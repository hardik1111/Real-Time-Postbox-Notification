\label{chapter:Future Scope}
Although the prototype of the smart postbox works as per expectation, the ability of its usefulness can be enhanced further with some additional module implementation and careful coding logic.

First of all, at this point, we have only one way of notifying the user which is e-mail, but we can also implement an SMS service, which can be added via the GSM module or third-party services can be also used.

In the upcoming days, our proposed system will be implemented with a GUI interface inform of application and website, which will show all the related data such as time of incoming post and counter of posts. 

Furthermore, our algorithm is very naive, sometimes it cannot comprehend the availability of two posts at the same time, it gives faulty readings. Along with there are many real-life scenarios that we have overlooked such as, if a user only takes out only one of many posts, this will also create distance change and the system will give new post notification without having one. 

Lastly, one issue is that we have used a NodeMCU module for connection which requires the internet for working properly. This problem could be mitigated by using the GSM module. While this set of technology has its own problem It fulfills the current objectives. 